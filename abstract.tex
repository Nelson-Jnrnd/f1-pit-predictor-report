% Francais
L'objectif d'une course automobile est de parcourir un nombre de tours sur un circuit le plus
rapidement possible. De plus, dans de nombreuses catégories, les voitures peuvent s'arrêter au stand pour changer de pneus et/ ou se réapprovisionner en carburant.
La dégradation des pneus et le poids supplémentaire dû à la charge de carburant ont un impact majeur sur les temps au tour d'une voiture.
Décider quand et combien de fois rentrer dans les stands est donc crucial pour les écuries. À cela vienne s'ajouter les notions de neutralisation de la course pour cause d'incident.

Établir la stratégie optimale est un enjeu majeur pour les équipes de course. Il est donc nécessaire de développer un système d'analyse de données et de prédiction pour aider les écuries à prendre les meilleures décisions stratégiques pendant la course.
Ce système devra prendre en compte les données de la voiture, les performances des pneus et les conditions de la piste pour fournir des recommandations en temps réel aux équipes de course.

Dans ce travail, nous tentons de créer un tel système pour le championnat du monde de Formule 1.
Pour ce faire, nous entraînons des modèles de Machine Learning sur des données historiques de la dernière ère de la formule 1.

Un set de données a été établi à partir des données transmises par la Formule 1 pendant les courses.
Ces données ont été préparées de manière à former un ensemble complet et cohérent, permettant d'entraîner des modèles

Après l'expérimentation avec des réseaux de neurones et des réseaux de neurones récurrents,
nous avons réussi à développer des modèles capables de capturer les relations entre la stratégie de course et les données de la voiture.

Pour améliorer le système, nous pourrions inclure les facteurs météorologiques, développer une simulation de course pour améliorer la précision des prédictions
et tester l'apprentissage par renforcement pour développer des stratégies novatrices.