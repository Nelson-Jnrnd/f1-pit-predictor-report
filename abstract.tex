% Francais
L'objectif d'une course automobile est de parcourir un nombre de tours sur un circuit le plus
rapidement possible. De plus, dans de nombreuses catégories, les voitures peuvent s'arrêter au stand pour changer de pneus et/ ou se réapprovisionner en carburant.
La dégradation des pneus et le poids supplémentaire dû à la charge de carburant ont un impact majeur sur les temps au tour d'une voiture.
Décider quand et combien de fois rentrer dans les stands est donc crucial pour les écuries. À cela vienne s'ajouter les notions de neutralisation de la course pour cause d'incident.

Établir la stratégie optimale est un enjeu majeur pour les équipes de course. Il est donc nécessaire de développer un système d'analyse de données et de prédiction pour aider les écuries à prendre les meilleures décisions stratégiques pendant la course.
Ce système devra prendre en compte les données de la voiture, les performances des pneus et les conditions de la piste pour fournir des recommandations en temps réel aux équipes de course.

Dans ce travail, nous tentons de créer un tel système pour le championnat du monde de Formule 1.
Pour ce faire, nous entraînons des modèles de Machine Learning sur des données historiques de la dernière ère de la formule 1.

La première partie de ce travail se concentre sur la création d'un dataset à partir des données transmises par la Formule 1 pendant les courses.
Dans un second temps, nous entrainerons des modèles à suggérer une stratégie de course pour une voiture.